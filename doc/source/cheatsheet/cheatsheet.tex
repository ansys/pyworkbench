% Options for packages loaded elsewhere
\PassOptionsToPackage{unicode}{hyperref}
\PassOptionsToPackage{hyphens}{url}
\PassOptionsToPackage{dvipsnames,svgnames,x11names}{xcolor}
%
\documentclass[
  9pt,
  landscape]{article}

\usepackage{amsmath,amssymb}
\usepackage{iftex}
\ifPDFTeX
  \usepackage[T1]{fontenc}
  \usepackage[utf8]{inputenc}
  \usepackage{textcomp} % provide euro and other symbols
\else % if luatex or xetex
  \usepackage{unicode-math}
  \defaultfontfeatures{Scale=MatchLowercase}
  \defaultfontfeatures[\rmfamily]{Ligatures=TeX,Scale=1}
\fi
\usepackage{lmodern}
\ifPDFTeX\else  
    % xetex/luatex font selection
\fi
% Use upquote if available, for straight quotes in verbatim environments
\IfFileExists{upquote.sty}{\usepackage{upquote}}{}
\IfFileExists{microtype.sty}{% use microtype if available
  \usepackage[]{microtype}
  \UseMicrotypeSet[protrusion]{basicmath} % disable protrusion for tt fonts
}{}
\makeatletter
\@ifundefined{KOMAClassName}{% if non-KOMA class
  \IfFileExists{parskip.sty}{%
    \usepackage{parskip}
  }{% else
    \setlength{\parindent}{0pt}
    \setlength{\parskip}{6pt plus 2pt minus 1pt}}
}{% if KOMA class
  \KOMAoptions{parskip=half}}
\makeatother
\usepackage{xcolor}
\usepackage{listings}
\newcommand{\passthrough}[1]{#1}
\lstset{defaultdialect=[5.3]Lua}
\lstset{defaultdialect=[x86masm]Assembler}
\setlength{\emergencystretch}{3em} % prevent overfull lines
\setcounter{secnumdepth}{-\maxdimen} % remove section numbering
% Make \paragraph and \subparagraph free-standing
\makeatletter
\ifx\paragraph\undefined\else
  \let\oldparagraph\paragraph
  \renewcommand{\paragraph}{
    \@ifstar
      \xxxParagraphStar
      \xxxParagraphNoStar
  }
  \newcommand{\xxxParagraphStar}[1]{\oldparagraph*{#1}\mbox{}}
  \newcommand{\xxxParagraphNoStar}[1]{\oldparagraph{#1}\mbox{}}
\fi
\ifx\subparagraph\undefined\else
  \let\oldsubparagraph\subparagraph
  \renewcommand{\subparagraph}{
    \@ifstar
      \xxxSubParagraphStar
      \xxxSubParagraphNoStar
  }
  \newcommand{\xxxSubParagraphStar}[1]{\oldsubparagraph*{#1}\mbox{}}
  \newcommand{\xxxSubParagraphNoStar}[1]{\oldsubparagraph{#1}\mbox{}}
\fi
\makeatother


\providecommand{\tightlist}{%
  \setlength{\itemsep}{0pt}\setlength{\parskip}{0pt}}\usepackage{longtable,booktabs,array}
\usepackage{calc} % for calculating minipage widths
% Correct order of tables after \paragraph or \subparagraph
\usepackage{etoolbox}
\makeatletter
\patchcmd\longtable{\par}{\if@noskipsec\mbox{}\fi\par}{}{}
\makeatother
% Allow footnotes in longtable head/foot
\IfFileExists{footnotehyper.sty}{\usepackage{footnotehyper}}{\usepackage{footnote}}
\makesavenoteenv{longtable}
\usepackage{graphicx}
\makeatletter
\newsavebox\pandoc@box
\newcommand*\pandocbounded[1]{% scales image to fit in text height/width
  \sbox\pandoc@box{#1}%
  \Gscale@div\@tempa{\textheight}{\dimexpr\ht\pandoc@box+\dp\pandoc@box\relax}%
  \Gscale@div\@tempb{\linewidth}{\wd\pandoc@box}%
  \ifdim\@tempb\p@<\@tempa\p@\let\@tempa\@tempb\fi% select the smaller of both
  \ifdim\@tempa\p@<\p@\scalebox{\@tempa}{\usebox\pandoc@box}%
  \else\usebox{\pandoc@box}%
  \fi%
}
% Set default figure placement to htbp
\def\fps@figure{htbp}
\makeatother

\usepackage[landscape]{geometry}
\usepackage{amsmath,amsthm,amsfonts,amssymb}
\usepackage{color,graphicx}
\usepackage[bookmarks=false]{hyperref}
\usepackage{wrapfig}
\usepackage{eso-pic}
\usepackage{float}
\usepackage{tabularx}
\usepackage{multicol}
\usepackage{calc}
\usepackage{ifthen}
\usepackage{multirow}
\usepackage{listings}
\usepackage[x11names]{xcolor}
\usepackage[T1]{fontenc}
\usepackage[utf8]{inputenc}
\makeatletter
\@ifpackageloaded{caption}{}{\usepackage{caption}}
\AtBeginDocument{%
\ifdefined\contentsname
  \renewcommand*\contentsname{Table of contents}
\else
  \newcommand\contentsname{Table of contents}
\fi
\ifdefined\listfigurename
  \renewcommand*\listfigurename{List of Figures}
\else
  \newcommand\listfigurename{List of Figures}
\fi
\ifdefined\listtablename
  \renewcommand*\listtablename{List of Tables}
\else
  \newcommand\listtablename{List of Tables}
\fi
\ifdefined\figurename
  \renewcommand*\figurename{Figure}
\else
  \newcommand\figurename{Figure}
\fi
\ifdefined\tablename
  \renewcommand*\tablename{Table}
\else
  \newcommand\tablename{Table}
\fi
}
\newcommand*\listoflistings\lstlistoflistings
\AtBeginDocument{%
\renewcommand*\lstlistlistingname{List of Listings}
}
\makeatother
\makeatletter
\makeatother
\makeatletter
\@ifpackageloaded{caption}{}{\usepackage{caption}}
\@ifpackageloaded{subcaption}{}{\usepackage{subcaption}}
\makeatother

\usepackage{bookmark}

\IfFileExists{xurl.sty}{\usepackage{xurl}}{} % add URL line breaks if available
\urlstyle{same} % disable monospaced font for URLs
\hypersetup{
  pdftitle={PyWorkbench cheat sheet},
  colorlinks=true,
  linkcolor={blue},
  filecolor={Maroon},
  citecolor={Blue},
  urlcolor={Blue},
  pdfcreator={LaTeX via pandoc}}


% Copyright (C) 2024 ANSYS, Inc. and/or its affiliates.
% SPDX-License-Identifier: MIT
%
%
% Permission is hereby granted, free of charge, to any person obtaining a copy
% of this software and associated documentation files (the "Software"), to deal
% in the Software without restriction, including without limitation the rights
% to use, copy, modify, merge, publish, distribute, sublicense, and/or sell
% copies of the Software, and to permit persons to whom the Software is
% furnished to do so, subject to the following conditions:
%
% The above copyright notice and this permission notice shall be included in all
% copies or substantial portions of the Software.
%
% THE SOFTWARE IS PROVIDED "AS IS", WITHOUT WARRANTY OF ANY KIND, EXPRESS OR
% IMPLIED, INCLUDING BUT NOT LIMITED TO THE WARRANTIES OF MERCHANTABILITY,
% FITNESS FOR A PARTICULAR PURPOSE AND NONINFRINGEMENT. IN NO EVENT SHALL THE
% AUTHORS OR COPYRIGHT HOLDERS BE LIABLE FOR ANY CLAIM, DAMAGES OR OTHER
% LIABILITY, WHETHER IN AN ACTION OF CONTRACT, TORT OR OTHERWISE, ARISING FROM,
% OUT OF OR IN CONNECTION WITH THE SOFTWARE OR THE USE OR OTHER DEALINGS IN THE
% SOFTWARE.

\usepackage[landscape]{geometry}
\usepackage{amsmath,amsthm,amsfonts,amssymb}
\usepackage{color,graphicx}
\usepackage[bookmarks=false]{hyperref}
\usepackage{wrapfig}
\usepackage{eso-pic}
%add a monospace font
%\usepackage[defaultfam,tabular,lining]{montserrat}
\usepackage{float}
\usepackage{tabularx}
%\usepackage{blindtext}
%\usepackage{overpic}
\usepackage{multicol}
\usepackage{calc}
\usepackage{ifthen}
\usepackage{multirow}
\usepackage{listings}
\usepackage[x11names]{xcolor}
\usepackage[T1]{fontenc}
\usepackage[utf8]{inputenc}
%\usepackage[scale=0.95]{sourcecodepro}
%\usepackage{ragged2e}
%\usepackage[dvipsnames]{xcolor}

% This sets page margins to .5 inch if using letter paper, and to 1cm
% if using A4 paper. (This probably isn't strictly necessary.)
% If using another size paper, use default 1cm margins.
\ifthenelse{\lengthtest { \paperwidth = 11in}}
    { \geometry{top=.15in,left=.25in,right=.25in,bottom=.15in} }
    {\ifthenelse{ \lengthtest{ \paperwidth = 297mm}}
        {\geometry{top=1cm,left=1cm,right=1cm,bottom=1cm} }
        {\geometry{top=1cm,left=1cm,right=1cm,bottom=1cm} }
    }

% Turn off header and footer
\pagestyle{empty}
\usepackage{fontspec}
\usepackage{seqsplit}
\usepackage{sourcecodepro}

% Redefine section commands to use less space
\makeatletter
\renewcommand{\section}{\@startsection{section}{1}{0mm}%
                                {-1ex plus -.5ex minus -.2ex}%
                                {0.5ex plus .2ex}%
                                {\normalfont\Large\bfseries\includegraphics[height=\fontcharht\font`\S]{./_static/slash.png}\hspace{0.3em}}}
\renewcommand{\subsection}{\@startsection{subsection}{2}{0mm}%
                                {2ex plus -.5ex minus -.2ex}%
                                {0.5ex plus .2ex}%
                                {\normalfont\normalsize\bfseries}}
% \makeatother
\definecolor{silver}{RGB}{217,216,214}
\definecolor{gold}{RGB}{255,183,27}
\definecolor{steel}{RGB}{137,138,141}
\definecolor{lead}{RGB}{55,58,54}
\definecolor{bronze}{RGB}{200,146,17}
\definecolor{cblack}{RGB}{0,0,0}
\definecolor{codegreen}{rgb}{0,0.6,0}
\definecolor{codegray}{rgb}{0.5,0.5,0.5}
\definecolor{codepurple}{rgb}{0.58,0,0.82}
\definecolor{blue}{RGB}{57,114,161}
\definecolor{black}{RGB}{0,0,0}
\definecolor{shadecolor}{RGB}{217,216,214} % Set the desired background color
% renew if shaded is there
\lstdefinestyle{python_style}{
    backgroundcolor=\color{shadecolor}, % Set the background color
    commentstyle=\color{codegreen}, % Style for comments
    keywordstyle=\color{magenta}, % Style for keywords
    numberstyle=\tiny\color{codegray}, % Style for line numbers
    stringstyle=\color{codepurple}, % Style for strings
    breaklines=true, % Enable line breaking
    breakatwhitespace=true, % Break lines at whitespace
    % linewidth=\textwidth, % Set the width of the code block to match the text width
	basicstyle=\ttfamily\footnotesize, % Set the font style
    frame=single, % Add a frame around the code
    framesep=1pt, % Padding between the frame and the code
    xleftmargin=1pt, % Left margin for the code block (adjust as needed)
	rulecolor=\color{shadecolor},
    % xrightmargin=15pt, % Right margin for the code block (adjust as needed)
    showstringspaces=false, % Don't show spaces as special characters
	tabsize=1,
}
\lstset{style=python_style, upquote=true}

\def\code#1{\textit{}{#1}}

% Define BibTeX command
\def\BibTeX{{\rm B\kern-.05em{\sc i\kern-.025em b}\kern-.08em
    T\kern-.1667em\lower.7ex\hbox{E}\kern-.125emX}}

% Don't print section numbers
\setcounter{secnumdepth}{0}

\setlength{\parindent}{2pt}
\setlength{\parskip}{1pt plus 0.5ex}
\begin{document}
% Copyright (C) 2024 ANSYS, Inc. and/or its affiliates.
% SPDX-License-Identifier: MIT
%
%
% Permission is hereby granted, free of charge, to any person obtaining a copy
% of this software and associated documentation files (the "Software"), to deal
% in the Software without restriction, including without limitation the rights
% to use, copy, modify, merge, publish, distribute, sublicense, and/or sell
% copies of the Software, and to permit persons to whom the Software is
% furnished to do so, subject to the following conditions:
%
% The above copyright notice and this permission notice shall be included in all
% copies or substantial portions of the Software.
%
% THE SOFTWARE IS PROVIDED "AS IS", WITHOUT WARRANTY OF ANY KIND, EXPRESS OR
% IMPLIED, INCLUDING BUT NOT LIMITED TO THE WARRANTIES OF MERCHANTABILITY,
% FITNESS FOR A PARTICULAR PURPOSE AND NONINFRINGEMENT. IN NO EVENT SHALL THE
% AUTHORS OR COPYRIGHT HOLDERS BE LIABLE FOR ANY CLAIM, DAMAGES OR OTHER
% LIABILITY, WHETHER IN AN ACTION OF CONTRACT, TORT OR OTHERWISE, ARISING FROM,
% OUT OF OR IN CONNECTION WITH THE SOFTWARE OR THE USE OR OTHER DEALINGS IN THE
% SOFTWARE.

\raggedright
\footnotesize
\begin{center}
     \Huge{\textbf{PyWorkbench cheat sheet}} \\
\end{center}

\vspace{0.2cm}

\begin{center}
    \small{\textbf{Version: main}} \\
\end{center}
\AddToShipoutPicture*
	{\put(670,577.5){\includegraphics[height = 1.2cm]{./_static/ansys.png}}}
\AddToShipoutPictureBG*{\includegraphics[width=\paperwidth]{./_static/bground.png}}
\noindent\makebox[\linewidth]{\rule{\paperwidth}{2pt}}

\begin{multicols}{3}
% multicol parameters
% These lengths are set only within the two main columns
%\setlength{\columnseprule}{0.25pt}
\setlength{\premulticols}{1pt}
\setlength{\postmulticols}{1pt}
\setlength{\multicolsep}{1pt}
\setlength{\columnsep}{2pt}

\section{Connect PyWorkbench to Ansys Workbench from
Python}\label{connect-pyworkbench-to-ansys-workbench-from-python}

\subsection{Connect to a local Workbench
server}\label{connect-to-a-local-workbench-server}

Perform these steps to link PyWorkbench with a local session:

\begin{itemize}
\tightlist
\item
  Initiate Ansys Workbench.
\item
  Enter the \passthrough{\lstinline!StartServer()!} method in the
  Workbench command window.
\item
  Use the given port number to link PyWorkbench with the server.
\end{itemize}

\begin{lstlisting}[language=Python]
from ansys.workbench.core import launch_workbench
wb = launch_workbench()
\end{lstlisting}

\subsection{Connect to a remote Workbench
server}\label{connect-to-a-remote-workbench-server}

Execute the steps below to link PyWorkbench with a remote session:

\begin{lstlisting}[language=Python]
from ansys.workbench.core import connect_workbench
workbench = connect_workbench(port=port)
\end{lstlisting}

\section{Launch a Workbench server and start a
client}\label{launch-a-workbench-server-and-start-a-client}

You can launch a Workbench server and start a client through a Python
script on the client side.

This script initiates a server on a local Windows-based system:

\begin{lstlisting}[language=Python]
host = "server_machine_name_or_IP"
port = server_port_number
workbench = connect_workbench(host=host, port=port)
\end{lstlisting}

This script initiates a server on a remote Windows device using
appropriate user authentication:

\begin{lstlisting}[language=Python]
host = "server_machine_name_or_ip"
username = "your_username_on_server_machine"
password = "your_password_on_server_machine"
wb = launch_workbench(
    host=host, username=username, password=password
)
\end{lstlisting}

\section{Execute scripts on the Workbench
server}\label{execute-scripts-on-the-workbench-server}

These methods can be utilized to execute IronPython-based Workbench
scripts containing commands or queries, with the help of PyWorkbench:

\begin{itemize}
\tightlist
\item
  \passthrough{\lstinline!run\_script\_string()!}: Executes a script
  included within a string
\item
  \passthrough{\lstinline!run\_script\_file()!}: Executes a script file
  in the client's working directory
\end{itemize}

\subsubsection{\texorpdfstring{Use the \texttt{run\_script\_string()}
method:}{Use the run\_script\_string() method:}}\label{use-the-run_script_string-method}

\begin{lstlisting}[language=Python]
wbjn_template = """
import os
import json
import string
import os.path
work_dir = GetServerWorkingDirectory()
arg_ProjectArchive = os.path.join(work_dir, "MatDesigner.wbpz")
Unarchive(ArchivePath=arg_ProjectArchive,
    ProjectPath=GetAbsoluteUserPathName(work_dir + "wbpj\\MatDesigner.wbpj"),
    Overwrite=True)
    """
wb.run_script_string(wbjn_template)
\end{lstlisting}

\subsubsection{\texorpdfstring{Use the \texttt{run\_script\_file()}
method:}{Use the run\_script\_file() method:}}\label{use-the-run_script_file-method}

\begin{lstlisting}[language=Python]
wb.run_script_file("project_workflow.wbjn")
\end{lstlisting}

\subsubsection{Assign output to global
variable:}\label{assign-output-to-global-variable}

Assign necessary output from these methods to the global variable
\passthrough{\lstinline!wb\_script\_result!} as a JSON string. This
script returns all message summaries from the Workbench session:

\begin{lstlisting}[language=Python]
import json
messages = [m.Summary for m in GetMessages()]
wb_script_result = json.dumps(messages)
\end{lstlisting}

While executing, the following script displays
\passthrough{\lstinline!info!}, \passthrough{\lstinline!warning!}, and
\passthrough{\lstinline!error!} levels in the logger.

\begin{lstlisting}[language=Python]
wb.run_script_file("project_workflow.wbjn", log_level="info")
\end{lstlisting}

\section{Upload and download files}\label{upload-and-download-files}

Use the \passthrough{\lstinline!upload\_file()!} and
\passthrough{\lstinline!download\_file()!} methods to transfer data
files to and from the server.

Use the \passthrough{\lstinline!GetServerWorkingDirectory()!} query in
server-side scripts to obtain the server's operating directory.

This script uploads all PRT and AGDB files in the working directory from
the client to the server:

\begin{lstlisting}[language=Python]
wb.upload_file("model?.prt", "*.agdb")
\end{lstlisting}

This server-side script loads a geometry file into a new Workbench
system from the server's directory:

\begin{lstlisting}[language=Python]
wb.run_script_string(
    r"""import os
work_dir = GetServerWorkingDirectory()
geometry_file = os.path.join(work_dir, "my_geometry.agdb")
template = GetTemplate(TemplateName="Static Structural", Solver="ANSYS")
system = CreateSystemFromTemplate(Template=template,
Name="Static Structural (ANSYS)")
system.GetContainer( ComponentName="Geometry").SetFile( FilePath=geometry_file)
"""
)
\end{lstlisting}

This server-side script transfers a Mechanical solver output file to the
server's directory from Workbench:

\begin{lstlisting}[language=Python]
wb.run_script_string(
    r"""import os
import shutil
work_dir = GetServerWorkingDirectory()
mechanical_dir = mechanical.project_directory
out_file_src = os.path.join(mechanical_dir, "solve.out")
out_file_des = os.path.join(work_dir, "solve.out")
shutil.copyfile(out_file_src, out_file_des)
"""
)
\end{lstlisting}

This client script retrieves all .out files from the server's working
directory:

\begin{lstlisting}[language=Python]
wb.download_file("*.out")
\end{lstlisting}

Use the \passthrough{\lstinline!\{download\_project\_archive()\}!}
function to save, archive, and download the current Workbench project
from the server to the client:

\begin{lstlisting}[language=Python]
wb.download_project_archive( archive_name="my_project_archive")
\end{lstlisting}

\section{Initiate additional PyAnsys services for systems within a
Workbench
project}\label{initiate-additional-pyansys-services-for-systems-within-a-workbench-project}

\subsection{Link PyMechanical and
operate}\label{link-pymechanical-and-operate}

In a Workbench project, you can operate and link the PyMechanical
service from the same client machine.

This script creates a Mechanical system server side and then starts the
PyMechanical service and client:

\begin{lstlisting}[language=Python]
from ansys.mechanical.core import connect_to_mechanical

sys_name = wb.run_script_string(
    r"""import json
wb_script_result =
    json.dumps(GetTemplate(
    TemplateName="Static Structural (ANSYS)"
    ).CreateSystem().Name)
"""
)
server_port = wb.start_mechanical_server(
    system_name=sys_name
)
mechanical = connect_to_mechanical(
    ip="localhost", port=server_port
)
\end{lstlisting}

\subsection{Initiate the PyFluent
service}\label{initiate-the-pyfluent-service}

This script initiates the PyFluent service along with the client for a
Fluent system that was developed in Workbench:

\begin{lstlisting}[language=Python]
import ansys.fluent.core as pyfluent

sys_name = wb.run_script_string(
    r"""import json
wb_script_result =
    json.dumps(GetTemplate(
    TemplateName="FLUENT").CreateSystem().Name)
"""
)
server_info_file = wb.start_fluent_server(
    system_name=sys_name
)
fluent = pyfluent.connect_to_fluent(
    server_info_file_name=server_info_file
)
\end{lstlisting}

\subsection{Initiate the PySherlock
service}\label{initiate-the-pysherlock-service}

This script initiates the PySherlock service and its client for a
Sherlock system set up in Workbench:

\begin{lstlisting}[language=Python]
from ansys.sherlock.core import launcher as pysherlock

sys_name = wb.run_script_string(
    r"""import json
wb_script_result =
    json.dumps(GetTemplate(
    TemplateName="SherlockPre").CreateSystem().Name
    )
"""
)
server_port = wb.start_sherlock_server(
    system_name=sys_name
)
sherlock = pysherlock.connect_grpc_channel(
    port=server_port
)
\end{lstlisting}

\begin{itemize}
\tightlist
\item
  \href{https://workbench.docs.pyansys.com/version/stable/getting-started.html}{Getting
  started}
\item
  \href{https://workbench.docs.pyansys.com/version/stable/user-guide.html}{User
  guide}
\item
  \href{https://workbench.docs.pyansys.com/version/stable/examples.html}{Examples}
\item
  \href{https://workbench.docs.pyansys.com/version/stable/api/index.html}{API
  reference}
\end{itemize}



% Copyright (C) 2024 ANSYS, Inc. and/or its affiliates.
% SPDX-License-Identifier: MIT
%
%
% Permission is hereby granted, free of charge, to any person obtaining a copy
% of this software and associated documentation files (the "Software"), to deal
% in the Software without restriction, including without limitation the rights
% to use, copy, modify, merge, publish, distribute, sublicense, and/or sell
% copies of the Software, and to permit persons to whom the Software is
% furnished to do so, subject to the following conditions:
%
% The above copyright notice and this permission notice shall be included in all
% copies or substantial portions of the Software.
%
% THE SOFTWARE IS PROVIDED "AS IS", WITHOUT WARRANTY OF ANY KIND, EXPRESS OR
% IMPLIED, INCLUDING BUT NOT LIMITED TO THE WARRANTIES OF MERCHANTABILITY,
% FITNESS FOR A PARTICULAR PURPOSE AND NONINFRINGEMENT. IN NO EVENT SHALL THE
% AUTHORS OR COPYRIGHT HOLDERS BE LIABLE FOR ANY CLAIM, DAMAGES OR OTHER
% LIABILITY, WHETHER IN AN ACTION OF CONTRACT, TORT OR OTHERWISE, ARISING FROM,
% OUT OF OR IN CONNECTION WITH THE SOFTWARE OR THE USE OR OTHER DEALINGS IN THE
% SOFTWARE.

\end{multicols}
\AtEndDocument{
    \noindent\makebox[\linewidth]{\rule{\paperwidth}{2pt}}
\begin{center}
PyWorkbench
\includegraphics[height=\fontcharht\font`\S]{./_static/slash.png} \href{https://workbench.docs.pyansys.com/version/stable/}{{Documentation}}
\includegraphics[height=\fontcharht\font`\S]{./_static/slash.png} \href{https://workbench.docs.pyansys.com/version/stable/getting\_started/index.html}{{Getting
started}}
\includegraphics[height=\fontcharht\font`\S]{./_static/slash.png} \href{https://workbench.docs.pyansys.com/version/stable/examples.html}{{Examples}}
\includegraphics[height=\fontcharht\font`\S]{./_static/slash.png} \href{https://workbench.docs.pyansys.com/version/stable/api/index.html}{{API
reference}}
\includegraphics[height=\fontcharht\font`\S]{./_static/slash.png} \href{https://workbench.docs.pyansys.com/version/stable/getting\_started/faq.html}{{FAQ}}
\includegraphics[height=\fontcharht\font`\S]{./_static/slash.png} \href{https://github.com/ansys/pyworkbench/discussions}{{Discussions}}
\includegraphics[height=\fontcharht\font`\S]{./_static/slash.png} \href{https://github.com/ansys/pyworkbench/issues}{{Issues}}
\end{center}
}
\end{document}
